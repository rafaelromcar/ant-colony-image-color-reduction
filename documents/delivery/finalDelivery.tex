\documentclass{pid}
\usepackage{times}
\usepackage{graphicx}
\usepackage{url}
\usepackage[spanish]{babel}
\usepackage[latin1]{inputenc}

\begin{document}
\begin{frontmatter}

\title{Reducci�n del color mediante colonias de hormigas}
\author{Neira Ayuso, �lvaro}\\
\author{Rivas M�ndez, Ignacio}\\
\author{Romero Carmona, Rafael}\\

\maketitle

\begin{abstract}
\noindent

\end{abstract}

\begin{keyword} palabras clave (m�s o menos, cinco palabras que clasifiquen el trabajo) \sep pid \sep instrucciones\sep trabajo \sep imagen digital \end{keyword}
\end{frontmatter}

\section{Introducci�n}


\section{Planteamiento te�rico}

\subsection{EJEMPLO SUBSECCION}

ESTO DE AQU� ES PARA LISTAR DOCUMENTOS Y MOSTRARLOS DE MANERA CONCRETA. MIRA EL EJEMPO.PDF y EJEMPLO.TEX PARA VER TODO BIEN.
\begin{itemize}
\item \url{Documentacion.zip}.

Documentacion.zip debe contener el fichero \url{.pdf} correspondiente a la documentaci�n (tambi�n el \url{.tex} y las figuras \url{.png}, si se ha escrito en Latex). Adem�s, se debe incluir un resumen en un archivo tipo texto \url{.txt} que describa lo m�s fielmente posible el trabajo realizado (no el propuesto inicialmente, que podr�a ser distinto), de m�s de 70 palabras y menos de 250.     
 
\item \url{Codigo.zip}. Debe contener todo el c�digo fuente utilizado.
\item \url {Ejecutable.zip}. Debe contener la aplicaci�n junto con todos los archivos necesarios para su ejecuci�n de forma que no d� errores de compilaci�n. 
Si es necesario, debe contener un leeme.txt con instrucciones para ejecutar la aplicaci�n.
Se debe incluir una {\bf carpeta con im�genes de muestra}. 

\end{itemize}

\section{Resoluci�n Pr�ctica}
OTRA FORMA DE METER ITEMS EN PAN LISTA DE ARRIBA A ABAJO

\begin{enumerate}
\item   Resumen.
\item  Introducci�n.
\item   Planteamiento te�rico.
\item   Resoluci�n pr�ctica.
\item Experimentaci�n. 
\item    Manual de usuario.
\item Conclusiones.
\item Referencias.
\item Anexo: Tabla de tiempos. 
\end{enumerate}

\section{Experimentaci�n}


\section{Manual de usuario}


\section{Conclusiones}



\section{Referencias}

Las referencias se citan as�:
bla, bla \cite{clave:revista}, bla, bla \cite{clave:libro}. La bibliograf�a
debe seguir el estilo de este documento. A continuaci�n aparecen dos ejemplos de referencias bibliogr�ficas: un art�culo en una revista y un libro.

\begin{verbatim}
\begin{thebibliography}{9}
\bibitem{clave:revista}
Y. O. Mismo, ``Alg�n trabajo relacionado", \emph{Publicaci�n Peri�dica}, 
Vol. 17, pp. 1-100, 1997.

\bibitem{clave:libro}
U. N. Experto, \emph{Un libro que escribi�}, Editorial, 1996.
\end{thebibliography}
\end{verbatim}

El resultado de estos ejemplos puede verse a continuaci�n, con las referencias ordenadas alfab�ticamente por autores.



\begin{thebibliography}{10}
\bibitem{clave:libro}
U. N. Experto, \emph{Un libro que escribi�},
Editorial, 1996.

\bibitem{clave:revista}
Y. O. Mismo,
``Alg�n trabajo relacionado'',
\emph{Publicaci�n Peri�dica}, Vol. 17, pp. 1-100, 1997.

\bibitem{clave:url}
\url{http://laojamientos.us.es/gotocma/pid}


\end{thebibliography}
\newpage
\noindent {\bf Anexo I: Tabla de tiempos}


Se debe justificar el trabajo realizado por cada componente del grupo, comentando el tiempo total que cada miembro del grupo ha dedicado al trabajo (lo que puede implicar diferencia de notas obtenidas por los distintos miembros del grupo). El trabajo realizado debe ser de {\bf 78 horas por alumno}. Adem�s, debe haber un plan de trabajo detallado. Para esto �ltimo, se puede usar la tabla siguiente o bien documentos generados por la herramienta de gesti�n de proyectos que se use, como Projetsii o Microsoft Project, por ejemplo.
$$\begin{array}{|c|c|c|c|c|c|}
\hline 
\mbox{ Fecha de la
reuni�n }
& \mbox{ Inicio }
& \mbox{ Fin }
&\mbox{ Tiempo
total
empleado }
&	\mbox{ 
Miembros
del grupo
reunidos }
&	\mbox{ Actividad }\\\hline
\mbox{ }&\mbox{}&\mbox{ }&\mbox{}&\mbox{ }&\mbox{}\\
\mbox{ }&\mbox{}&\mbox{ }&\mbox{}&\mbox{ }&\mbox{}\\\hline\end{array}$$

\end{document}
