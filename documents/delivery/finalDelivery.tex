\documentclass{pid}
\usepackage{times}
\usepackage{graphicx}
\usepackage{url}
\usepackage[spanish]{babel}
\usepackage[latin1]{inputenc}

\begin{document}
\begin{frontmatter}

\title{Reducción del color mediante colonias de hormigas}
\author{Neira Ayuso, Álvaro}\\
\author{Rivas Méndez, Ignacio}\\
\author{Romero Carmona, Rafael}\\

\maketitle

\begin{abstract}
\noindent

\end{abstract}

\begin{keyword} palabras clave (más o menos, cinco palabras que clasifiquen el trabajo) \sep \sep \sep \sep \end{keyword}
\end{frontmatter}

\section{Introducción}

Durante el siguiente artículo se explicará un método para la reducción del número de colores en imágenes usando un algoritmo basado en colonias de hormigas para la generación de la nueva paleta de color.\\

Los algoritmos de reducción de colores tienen dos fases básicas, la primera en la que se calcula una paleta de color nueva y en la segunda en la que se intercambian los colores de la imagen por los de esta paleta.\\

En nuestro caso se puede dividir el primer paso en otros dos, clasificación de los colores de la imagen en conjuntos y elección del color mas representativo de cada conjunto.\\

Es precisamente en la clasificación de los colores de la imagen en la que se aprovecha la eficiencia de los algoritmos de clasificación basados en colonias de hormigas. Estos algoritmo bioinspirados de inteligencia artificial se distinguen por dar resultados cercanos a la solución óptima de forma rápida a problemas computacionalmente duros.\\

Hay dos tipos de algoritmos basados en colonias de hormigas, y su diferencia principal radica en que usen o no feromonas. Estas feromonas se encargan de que una hormiga tenga mas posibilidades de seguir el camino que han seguido una hormiga anterior. \\

En nuestro caso al ser un algoritmo de clasificación no son necesarias estas feromonas ya que la función de las hormigas en recoger un elemento y moverlo de una posición a otra siguiendo un camino, que sera descrito por los elementos de la imagen, no por la ruta seguida por las hormigas anteriores. \\

Este método de clasificación basado en colonia de hormigas sin uso de feromonas fue desarrollado por Handl y Meyer en 2002 y será la base de nuestro algoritmo, aunque se le introducirá un elemento que tiene el método de Lumer y Faieta en 1994, la memoria de la hormiga, que ayudará a reducir el tiempo de convergencia del algoritmo.\\

Uniendo estos elementos se probará que se consiguen mejores resultados de mayor calidad que con otros algoritmos de reducción de colores existentes como son el método de Bing, el C-means, Centre-cut y Octree\\

\section{Planteamiento teórico}
A continuación pasaremos a explicar el fundamento teórico del algoritmo para resolver el problema de reducción de colores en una imagen que estamos presentando.\\

Este algoritmo divide el problema de la reducción de colores en tres funcionalidades diferentes: hacer una clasificación de una cantidad significativa de píxeles de la imagen, elegir el mas representativos de cada uno de los grupos creados y sustituir los colores de la imagen con estos píxeles seleccionados.\\

Para llevar esto a cabo, utiliza una variación del algoritmo de clasificación de datos basado en colonia de hormigas de Handl, que realiza la primera de las funcionalidades para datos conjuntos de datos generales. Estas variaciones, además de añadir las dos nuevas funcionalidades, también toma en cuenta las características propias de las imágenes, como son la similitud entre píxeles vecinos y las características específicas de estos.\\

El algoritmo de Handl tiene 3 fases. En la primera hace un primer preprocesamiento para obtener un conjunto de clases (conjuntos pequeños de datos similares). Después coge datos al azar del conjunto completo dado y los coloca en las clases más similares a ellos. De esta forma crea un conjunto de pequeñas clases de objetos similares con las que, tras un proceso de unión y refinamiento, se obtienen la clasificación buscada.\\

\subsection{Clasificación de píxles}
Esta es la funcionalidad en la que destaca nuestro método ya que el uso de inteligencia artificial a través del método de la colonia de hormigas otorga un resultado de mayor calidad que otros métodos como veremos más adelante.i\\

Se desarrolla de forma muy parecida a la descrita anteriormente en el algoritmo de Handl, pero con algunas variaciones tanto en la estructura de los datos como en el procesado. \\

Para nuestro caso específico se cambian dos estructuras usadas en el algoritmo base, la primera es el concepto de clase, conjuntos de datos con los que se comparan los píxeles seleccionados, que en nuestro caso será sustituido por la memoria de la hormiga, siendo esta un conjunto del mismo tamaño que la paleta de colores deseada que contendrá píxeles y las posiciones de los mismos. La otra variación en la estructura de datos se encuentra en la imagen que se procesa, ya que en nuestro caso cada posición de la imagen podrá estar vacía, contener un píxel o una pila de píxeles.\\

Las divergencias en el procesado son sencillas, en nuestro algoritmo se hace un doble procesado, comparando en primer lugar el píxel con la memoria de la hormiga y en caso de que no se encuentre ningún elemento similar en esta, se procede a buscar en las posiciones cercanas al píxel, haciendo que la hormiga se desplace por la imagen buscando un píxel parecido al suyo. \\

De esta forma nuestro algoritmo de clasificación repite N veces los siguientes pasos:\\
\begin{enumerate}
    \item Elige un píxel al azar.\\
    \item Busca un elemento parecido en la memoria de la hormiga\\
    \item SI lo encuentra coloca el píxel seleccionado en la pila que se encuentra en la posición de este elemento en la imagen\\
    \item SI NO, compara el píxel seleccionado con los elementos vecinos a él y si encuentra uno similar se añade a la pila de este.\\
\end{enumerate}
De esta forma, tras iterar un número determinado de veces se podrá encontrar la clasificación buscada en las posiciones a las que apuntan los elementos de la memoria de la hormiga.\\

\subsection{Elección y Sustitución de los colores a  partir de los clusters}
Una vez clasificados los colores por el método anterior, los otros dos pasos son sencillos y computacionalmente muy rápidos. Por esta razón hemos considerado que durante la experimentación usaremos diversos métodos estadísticos para llevar a cabo estas funcionalidades.\\

En el primer caso debemos encontrar el píxel más significativo de un conjunto dado, para ello realizaremos experimentos usando distintas medidas de tendencia general, como la mediana y varios tipos de medias.\\

En el caso de la sustitución de colores no será necesaria este método tan riguroso para encontrar la medida, ya que esta funcionalidad requiere una función que calcule la divergencia entre dos píxeles diferentes y para ello podemos utilizar la misma función de comparación de píxeles que se usa durante el algoritmo.\\

\subsection{Detalles sobre los planteamientos matemáticos}
Para la realización del algoritmo descrito anteriormente es necesaria la especificación matemática de algunos conceptos citados. Estos conceptos son divergencia entre dos píxeles y distancia entre un píxel y sus vecinos.\\

Los píxeles,en nuestro caso, están compuesto de tres componentes (rojo, verde y azul), por lo que las funciones mas lógicas lógicas para utilizar son la distancia Euclídea y la distancia L1.\\

Distancia Euclídea entre el píxel i y el j → \\ 
Distancia L1 entre el píxel i y el j  → \\

Estas opciones fueron decididas por lo bien que representan la distancia entre componentes de un vector y por la simplicidad de su cálculo, ya que esta será la función que más veces llevaremos a cabo durante la ejecución de nuestro algoritmo.\\

Basándonos en un análisis preliminar de estas funciones concluimos en que la distancia L1 produce resultados con menos distorsión que los encontrados con la distancia Euclídea.\\

Para calcular la distancia ente un píxel y sus vecinos tendremos en cuenta la divergencia entre el este píxel y cada uno de sus vecinos,  pero también tenemos que tener en cuenta que se dará la situación en la que no existan alguno de los vecinos, ya anteriormente se han podido recoger estos píxeles y soltar en otra posición.\\

Distancia a los vecinos del píxel i → \\

Los valores delta y alfa representan respectivamente el radio de percepción (cantidad de píxeles máximo que se comparan en cada dirección) y la influencia de la función divergencia.\\

Estos parámetros se especificarán a un valor óptimo durante la fase de experimentación.\\
\section{Resolución Práctica}

\section{Experimentación}

\section{Manual de usuario}

\section{Conclusiones}

\section{Referencias}

\newpage
\noindent {\bf Anexo I: Tabla de tiempos}


Se debe justificar el trabajo realizado por cada componente del grupo, comentando el tiempo total que cada miembro del grupo ha dedicado al trabajo (lo que puede implicar diferencia de notas obtenidas por los distintos miembros del grupo). El trabajo realizado debe ser de {\bf 78 horas por alumno}. Además, debe haber un plan de trabajo detallado. Para esto último, se puede usar la tabla siguiente o bien documentos generados por la herramienta de gestión de proyectos que se use, como Projetsii o Microsoft Project, por ejemplo.
$$\begin{array}{|c|c|c|c|c|c|}
\hline 
\mbox{ Fecha de la
reunión }
& \mbox{ Inicio }
& \mbox{ Fin }
&\mbox{ Tiempo
total
empleado }
&	\mbox{ 
Miembros
del grupo
reunidos }
&	\mbox{ Actividad }\\\hline
\mbox{ }&\mbox{}&\mbox{ }&\mbox{}&\mbox{ }&\mbox{}\\
\mbox{ }&\mbox{}&\mbox{ }&\mbox{}&\mbox{ }&\mbox{}\\\hline\end{array}$$

\end{document}
